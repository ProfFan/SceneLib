%   Scene: software for sequential localisation and map-building
%
%   models.tex
%   Copyright (C) 2000 Andrew Davison
%   ajd@robots.ox.ac.uk
%   Scene home page: http://www.robots.ox.ac.uk/~ajd/Scene/
%
%   This library is free software; you can redistribute it and/or
%   modify it under the terms of the GNU Lesser General Public
%   License as published by the Free Software Foundation; either
%   version 2.1 of the License, or (at your option) any later version.
%
%   This library is distributed in the hope that it will be useful,
%   but WITHOUT ANY WARRANTY; without even the implied warranty of
%   MERCHANTABILITY or FITNESS FOR A PARTICULAR PURPOSE.  See the GNU
%   Lesser General Public License for more details.
%
%   You should have received a copy of the GNU Lesser General Public
%   License along with this library; if not, write to the Free Software
%   Foundation, Inc., 59 Temple Place, Suite 330, Boston, MA  02111-1307  USA

%   Documentation on models and state representation in Scene

\documentclass{article}
%
\usepackage{epsfig}
\usepackage{makeidx}  % allows for index generation
%


\begin{document}

\unitlength=1mm
\newcommand{\ba}{\begin{array}}
\newcommand{\ea}{\end{array}}
\newcommand{\bc}{\begin{center}}
\newcommand{\ec}{\end{center}}
\newcommand{\be}{\begin{enumerate}}
\newcommand{\ee}{\end{enumerate}}
\newcommand{\bea}{\begin{eqnarray}}
\newcommand{\eea}{\end{eqnarray}}
\newcommand{\beas}{\begin{eqnarray*}}
\newcommand{\eeas}{\end{eqnarray*}}
\newcommand{\beq}{\begin{equation}}
\newcommand{\eeq}{\end{equation}}
\newcommand{\bfig}{\begin{figure}}
\newcommand{\efig}{\end{figure}}
\newcommand{\bi}{\begin{itemize}}
\newcommand{\ei}{\end{itemize}}
\newcommand{\bpic}{\begin{picture}}
\newcommand{\epic}{\end{picture}}
\newcommand{\btabular}{\begin{tabular}}
\newcommand{\etabular}{\end{tabular}}
\newcommand{\btable}{\begin{table}}
\newcommand{\etable}{\end{table}}

% Black and white \bs and \es
%\newcommand{\bs}[1]{\begin{slide}{}{\tiny #1} \\
%                    \rule[6mm]{170mm}{0.7mm}}
%\newcommand{\es}{\vfill
%                 \rule[-6mm]{170mm}{0.7mm} \\
%                 {\tiny 
%		  \hfill S-\theslide}
%                 \end{slide}}
%\newcommand{\Ltitle}[1]{\bc {\Large\bf \vspace{-15mm} #1} \ec}
%\newcommand{\ltitle}[1]{\bc {\large\bf \vspace{-15mm} #1} \ec}

% Colour versions of \bs and \es
\newcommand{\bs}[1]{\begin{slide}{}\redw{{\tiny #1}} \\
                    \rule[6mm]{170mm}{0.7mm}}
\newcommand{\es}{\vfill
                 \rule[-6mm]{170mm}{0.7mm} \\
                 \redw{{\tiny 
		  \hfill S-\theslide}}
                 \end{slide}}
\newcommand{\Ltitle}[1]{\bc {\Large\bf \vspace{-15mm} #1} \ec}
\newcommand{\ltitle}[1]{\bc {\large\bf \vspace{-15mm} #1} \ec}

\newcommand{\ruleoff}{\mbox{\\\rule{\columnwidth}{0.1mm}}}

% Originally \newcommand{\dsp}{\setlength{\baselineskip}{22truept}}
%\newcommand{\dsp}{\setlength{\baselineskip}{21truept}}
\newcommand{\matxx}[1]{{\tt #1}}
\newcommand{\vecXX}[1]{{\bf #1}}
\newcommand{\icap}[1]{{\caption{\em #1}}}

\def \nor {\normalsize}




\def \crossprod {\times}
\def \dotprod {\bullet}
\def \etal {{\em et al.}}
\def \grad {\mbox{\boldmath $\nabla$}}
\def \gradE {\mbox{{\boldmath $\nabla$}} E}
\def \half {\frac{1}{2}}
\def \matxTx {[{\tt t}_\times]}
\def \matxminusTx {[{\tt -t}_\times]}
\def \trans {{\top}}
\def \cosec {\mbox{cosec}}

\def \ahat {{\hat{a}}}
\def \bhat {{\hat{b}}}
\def \chat {{\hat{c}}}
\def \dhat {{\hat{d}}}
\def \ehat {{\hat{e}}}
\def \fhat {{\hat{f}}}
\def \ghat {{\hat{g}}}
\def \hhat {{\hat{h}}}
\def \ihat {{\hat{i}}}
\def \jhat {{\hat{j}}}
\def \khat {{\hat{k}}}
\def \lhat {{\hat{l}}}
\def \mhat {{\hat{m}}}
\def \nhat {{\hat{n}}}
\def \ohat {{\hat{o}}}
\def \phat {{\hat{p}}}
\def \qhat {{\hat{q}}}
\def \rhat {{\hat{r}}}
\def \shat {{\hat{s}}}
\def \that {{\hat{t}}}
\def \uhat {{\hat{u}}}
\def \vhat {{\hat{v}}}
\def \what {{\hat{w}}}
\def \xhat {{\hat{x}}}
\def \yhat {{\hat{y}}}
\def \zhat {{\hat{z}}}

\def \Ahat {{\hat{A}}}
\def \Bhat {{\hat{B}}}
\def \Chat {{\hat{C}}}
\def \Dhat {{\hat{D}}}
\def \Ehat {{\hat{E}}}
\def \Fhat {{\hat{F}}}
\def \Ghat {{\hat{G}}}
\def \Hhat {{\hat{H}}}
\def \Ihat {{\hat{I}}}
\def \Jhat {{\hat{J}}}
\def \Khat {{\hat{K}}}
\def \Lhat {{\hat{L}}}
\def \Mhat {{\hat{M}}}
\def \Nhat {{\Hat{N}}}
\def \Ohat {{\hat{O}}}
\def \Phat {{\hat{P}}}
\def \Qhat {{\hat{Q}}}
\def \Rhat {{\hat{R}}}
\def \Shat {{\hat{S}}}
\def \That {{\hat{T}}}
\def \Uhat {{\hat{U}}}
\def \Vhat {{\hat{V}}}
\def \What {{\hat{W}}}
\def \Xhat {{\hat{X}}}
\def \Yhat {{\hat{Y}}}
\def \Zhat {{\hat{Z}}}

\def \abar {{\bar{a}}}
\def \bbar {{\bar{b}}}
\def \cbar {{\bar{c}}}
\def \dbar {{\bar{d}}}
\def \ebar {{\bar{e}}}
\def \fbar {{\bar{f}}}
\def \gbar {{\bar{g}}}
\def \hbar {{\bar{h}}}
\def \ibar {{\bar{i}}}
\def \jbar {{\bar{j}}}
\def \kbar {{\bar{k}}}
\def \lbar {{\bar{l}}}
\def \mbar {{\bar{m}}}
\def \nbar {{\bar{n}}}
\def \obar {{\bar{o}}}
\def \pbar {{\bar{p}}}
\def \qbar {{\bar{q}}}
\def \rbar {{\bar{r}}}
\def \sbar {{\bar{s}}}
\def \tbar {{\bar{t}}}
\def \ubar {{\bar{u}}}
\def \vbar {{\bar{v}}}
\def \wbar {{\bar{w}}}
\def \xbar {{\bar{x}}}
\def \ybar {{\bar{y}}}
\def \zbar {{\bar{z}}}

\def \vecnought {{\vecXX{0}}}
\def \veca {{\vecXX{a}}}
\def \vecb {{\vecXX{b}}}
\def \vecc {{\vecXX{c}}}
\def \vecd {{\vecXX{d}}}
\def \vece {{\vecXX{e}}}
\def \vecf {{\vecXX{f}}}
\def \vecg {{\vecXX{g}}}
\def \vech {{\vecXX{h}}}
\def \veci {{\vecXX{i}}}
\def \vecj {{\vecXX{j}}}
\def \veck {{\vecXX{k}}}
\def \vecl {{\vecXX{l}}}
\def \vecm {{\vecXX{m}}}
\def \vecn {{\vecXX{n}}}
\def \veco {{\vecXX{o}}}
\def \vecp {{\vecXX{p}}}
\def \vecq {{\vecXX{q}}}
\def \vecr {{\vecXX{r}}}
\def \vecs {{\vecXX{s}}}
\def \vect {{\vecXX{t}}}
\def \vecu {{\vecXX{u}}}
\def \vecv {{\vecXX{v}}}
\def \vecw {{\vecXX{w}}}
\def \vecx {{\vecXX{x}}}
\def \vecy {{\vecXX{y}}}
\def \vecz {{\vecXX{z}}}

\def \vecA {{\vecXX{A}}}
\def \vecB {{\vecXX{B}}}
\def \vecC {{\vecXX{C}}}
\def \vecD {{\vecXX{D}}}
\def \vecE {{\vecXX{E}}}
\def \vecF {{\vecXX{F}}}
\def \vecG {{\vecXX{G}}}
\def \vecH {{\vecXX{H}}}
\def \vecI {{\vecXX{I}}}
\def \vecJ {{\vecXX{J}}}
\def \vecK {{\vecXX{K}}}
\def \vecL {{\vecXX{L}}}
\def \vecM {{\vecXX{M}}}
\def \vecN {{\vecXX{N}}}
\def \vecO {{\vecXX{O}}}
\def \vecP {{\vecXX{P}}}
\def \vecQ {{\vecXX{Q}}}
\def \vecR {{\vecXX{R}}}
\def \vecS {{\vecXX{S}}}
\def \vecT {{\vecXX{T}}}
\def \vecU {{\vecXX{U}}}
\def \vecV {{\vecXX{V}}}
\def \vecW {{\vecXX{W}}}
\def \vecX {{\vecXX{X}}}
\def \vecY {{\vecXX{Y}}}
\def \vecZ {{\vecXX{Z}}}
\def \vecnu {{\vecXX{\nu}}}

\def \Abar {{\bar{A}}}
\def \Bbar {{\bar{B}}}
\def \Cbar {{\bar{C}}}
\def \Dbar {{\bar{D}}}
\def \Ebar {{\bar{E}}}
\def \Fbar {{\bar{F}}}
\def \Gbar {{\bar{G}}}
\def \Hbar {{\bar{H}}}
\def \Ibar {{\bar{I}}}
\def \Jbar {{\bar{J}}}
\def \Kbar {{\bar{K}}}
\def \Lbar {{\bar{L}}}
\def \Mbar {{\bar{M}}}
\def \Nbar {{\bar{N}}}
\def \Obar {{\bar{O}}}
\def \Pbar {{\bar{P}}}
\def \Qbar {{\bar{Q}}}
\def \Rbar {{\bar{R}}}
\def \Sbar {{\bar{S}}}
\def \Tbar {{\bar{T}}}
\def \Ubar {{\bar{U}}}
\def \Vbar {{\bar{V}}}
\def \Wbar {{\bar{W}}}
\def \Xbar {{\bar{X}}}
\def \Ybar {{\bar{Y}}}
\def \Zbar {{\bar{Z}}}

\def \vecahat {\hat{\vecXX{a}}}
\def \vecbhat {\hat{\vecXX{b}}}
\def \vecchat {\hat{\vecXX{c}}}
\def \vecdhat {\hat{\vecXX{d}}}
\def \vecehat {\hat{\vecXX{e}}}
\def \vecfhat {\hat{\vecXX{f}}}
\def \vecghat {\hat{\vecXX{g}}}
\def \vechhat {\hat{\vecXX{h}}}
\def \vecihat {\hat{\vecXX{i}}}
\def \vecjhat {\hat{\vecXX{j}}}
\def \veckhat {\hat{\vecXX{k}}}
\def \veclhat {\hat{\vecXX{l}}}
\def \vecmhat {\hat{\vecXX{m}}}
\def \vecnhat {\hat{\vecXX{n}}}
\def \vecohat {\hat{\vecXX{o}}}
\def \vecphat {\hat{\vecXX{p}}}
\def \vecqhat {\hat{\vecXX{q}}}
\def \vecrhat {\hat{\vecXX{r}}}
\def \vecshat {\hat{\vecXX{s}}}
\def \vecthat {\hat{\vecXX{t}}}
\def \vecuhat {\hat{\vecXX{u}}}
\def \vecvhat {\hat{\vecXX{v}}}
\def \vecwhat {\hat{\vecXX{w}}}
\def \vecxhat {\hat{\vecXX{x}}}
\def \vecyhat {\hat{\vecXX{y}}}
\def \veczhat {\hat{\vecXX{z}}}

\def \vecabar {\bar{\vecXX{a}}}
\def \vecbbar {\bar{\vecXX{b}}}
\def \veccbar {\bar{\vecXX{c}}}
\def \vecdbar {\bar{\vecXX{d}}}
\def \vecebar {\bar{\vecXX{e}}}
\def \vecfbar {\bar{\vecXX{f}}}
\def \vecgbar {\bar{\vecXX{g}}}
\def \vechbar {\bar{\vecXX{h}}}
\def \vecibar {\bar{\vecXX{i}}}
\def \vecjbar {\bar{\vecXX{j}}}
\def \veckbar {\bar{\vecXX{k}}}
\def \veclbar {\bar{\vecXX{l}}}
\def \vecmbar {\bar{\vecXX{m}}}
\def \vecnbar {\bar{\vecXX{n}}}
\def \vecobar {\bar{\vecXX{o}}}
\def \vecpbar {\bar{\vecXX{p}}}
\def \vecqbar {\bar{\vecXX{q}}}
\def \vecrbar {\bar{\vecXX{r}}}
\def \vecsbar {\bar{\vecXX{s}}}
\def \vectbar {\bar{\vecXX{t}}}
\def \vecubar {\bar{\vecXX{u}}}
\def \vecvbar {\bar{\vecXX{v}}}
\def \vecwbar {\bar{\vecXX{w}}}
\def \vecxbar {\bar{\vecXX{x}}}
\def \vecybar {\bar{\vecXX{y}}}
\def \veczbar {\bar{\vecXX{z}}}

\def \matA {{\matxx{A}}}
\def \matB {{\matxx{B}}}
\def \matC {{\matxx{C}}}
\def \matD {{\matxx{D}}}
\def \matE {{\matxx{E}}}
\def \matF {{\matxx{F}}}
\def \matG {{\matxx{G}}}
\def \matH {{\matxx{H}}}
\def \matI {{\matxx{I}}}
\def \matJ {{\matxx{J}}}
\def \matK {{\matxx{K}}}
\def \matL {{\matxx{L}}}
\def \matM {{\matxx{M}}}
\def \matN {{\matxx{N}}}
\def \matO {{\matxx{O}}}
\def \matP {{\matxx{P}}}
\def \matQ {{\matxx{Q}}}
\def \matR {{\matxx{R}}}
\def \matS {{\matxx{S}}}
\def \matT {{\matxx{T}}}
\def \matU {{\matxx{U}}}
\def \matV {{\matxx{V}}}
\def \matW {{\matxx{W}}}
\def \matX {{\matxx{X}}}
\def \matY {{\matxx{Y}}}
\def \matZ {{\matxx{Z}}}
\def \matnought {{\matxx{0}}}

\def \matAhat {\hat{\matxx{A}}}
\def \matBhat {\hat{\matxx{B}}}
\def \matChat {\hat{\matxx{C}}}
\def \matDhat {\hat{\matxx{D}}}
\def \matEhat {\hat{\matxx{E}}}
\def \matFhat {\hat{\matxx{F}}}
\def \matGhat {\hat{\matxx{G}}}
\def \matHhat {\hat{\matxx{H}}}
\def \matIhat {\hat{\matxx{I}}}
\def \matJhat {\hat{\matxx{J}}}
\def \matKhat {\hat{\matxx{K}}}
\def \matLhat {\hat{\matxx{L}}}
\def \matMhat {\hat{\matxx{M}}}
\def \matNhat {\hat{\matxx{N}}}
\def \matOhat {\hat{\matxx{O}}}
\def \matPhat {\hat{\matxx{P}}}
\def \matQhat {\hat{\matxx{Q}}}
\def \matRhat {\hat{\matxx{R}}}
\def \matShat {\hat{\matxx{S}}}
\def \matThat {\hat{\matxx{T}}}
\def \matUhat {\hat{\matxx{U}}}
\def \matVhat {\hat{\matxx{V}}}
\def \matWhat {\hat{\matxx{W}}}
\def \matXhat {\hat{\matxx{X}}}
\def \matYhat {\hat{\matxx{Y}}}
\def \matZhat {\hat{\matxx{Z}}}

\def \matAbar {\bar{\matxx{A}}}
\def \matBbar {\bar{\matxx{B}}}
\def \matCbar {\bar{\matxx{C}}}
\def \matDbar {\bar{\matxx{D}}}
\def \matEbar {\bar{\matxx{E}}}
\def \matFbar {\bar{\matxx{F}}}
\def \matGbar {\bar{\matxx{G}}}
\def \matHbar {\bar{\matxx{H}}}
\def \matIbar {\bar{\matxx{I}}}
\def \matJbar {\bar{\matxx{J}}}
\def \matKbar {\bar{\matxx{K}}}
\def \matLbar {\bar{\matxx{L}}}
\def \matMbar {\bar{\matxx{M}}}
\def \matNbar {\bar{\matxx{N}}}
\def \matObar {\bar{\matxx{O}}}
\def \matPbar {\bar{\matxx{P}}}
\def \matQbar {\bar{\matxx{Q}}}
\def \matRbar {\bar{\matxx{R}}}
\def \matSbar {\bar{\matxx{S}}}
\def \matTbar {\bar{\matxx{T}}}
\def \matUbar {\bar{\matxx{U}}}
\def \matVbar {\bar{\matxx{V}}}
\def \matWbar {\bar{\matxx{W}}}
\def \matXbar {\bar{\matxx{X}}}
\def \matYbar {\bar{\matxx{Y}}}
\def \matZbar {\bar{\matxx{Z}}}

\def \vecrdot {\dot{\bf r}}
\def \vecth {{\bf \Theta}}
\def \vecthdot {\dot{\bf \Theta}}
\def \vecOmega {{\bf \Omega}}
\def \Ox {\Omega_x}
\def \Oy {\Omega_y}
\def \Oz {\Omega_z}
\def \meanrdot {\langle\vecrdot_s\rangle}
\def \meanr {\langle\vecr_s\rangle}

\def \curlrdot {{\rm curl}(\vecrdot)}
\def \divrdot  {{\rm div}(\vecrdot)}
\def \defrdot  {{\rm def}(\vecrdot)}

\def \apar {a_\parallel}
\def \aper {a_\perp}
\def \apars {a_\parallel^*}
\def \apers {a_\perp^*}

\def \deg {^{\circ}}

\def \dt {\delta t}
\def \Dt {\Delta t}

\def \dx {\delta x}
\def \Dx {\Delta x}

\def \dy {\delta y}
\def \Dy {\Delta y}

\def \t {\theta}
\def \tdot {\dot{\theta}}

\newcommand{\thdott}{\mbox{$\dot{\theta}_{t}$}}
\newcommand{\thdotmax}{\mbox{$\dot{\theta}_{pmax}$}}

\newcommand{\thddotmax}{\mbox{$\ddot{\theta}_{pmax}$}}
\newcommand{\mthdotmax}{\mbox{$|\dot{\theta}_{pmax}|$}}
\newcommand{\mthddotmax}{\mbox{$|\ddot{\theta}_{pmax}|$}}
\newcommand{\thti}{\mbox{$\theta_{ti}$}}
\newcommand{\thtzero}{\mbox{$\theta_{to}$}}
\newcommand{\thdottzero}{\mbox{$\dot{\theta}_{to}$}}
\newcommand{\thpzero}{\mbox{$\theta_{po}$}}
\newcommand{\thdotpzero}{\mbox{$\dot{\theta}_{po}$}}
\newcommand{\Tv}{\mbox{$T_v$}}
\newcommand{\Tu}{\mbox{$T_u$}}
\newcommand{\Tc}{\mbox{$T_c$}}
\newcommand{\Td}{\mbox{$T_d$}}






% AND SPECIALS

% A 3-element column vector \vecthree{arg1}{arg2}{arg3}
\newcommand{\vecthree}[3]{\left(\begin{array}{c}#1\\#2\\#3\end{array}\right)}

% A 4-element column vector \vecfour{arg1}{arg2}{arg3}{arg4}
\newcommand{\vecfour}[4]{\left(\begin{array}{c}#1\\#2\\#3\\#4\end{array}\right)}

% A 6-element column vector \vecsix{arg1}{arg2}{arg3}{arg4}{arg5}{arg6}
\newcommand{\vecsix}[6]{\left(\begin{array}{c}#1\\#2\\#3\\#4\\#5\\#6\end{array}\right)}

% A 2-element column vector \vectwo{arg1}{arg2}
\newcommand{\vectwo}[2]{\left(\begin{array}{c}#1\\#2\end{array}\right)}

% A 3-element column right-justified vector \vecthreeright{arg1}{arg2}{arg3}
\newcommand{\vecthreeright}[3]{\left(\begin{array}{r}#1\\#2\\#3\end{array}\right)}

% A 3-element row vector \rowthree{arg1}{arg2}{arg3}
% Trouble is this doesn't go to the right height next to 3*3 matrix
\newcommand{\rowthree}[3]{\left(\begin{array}{ccc}#1&#2&#3\end{array}\right)}

% A 4-element row vector \rowfour{arg1}{arg2}{arg3}{arg4}
\newcommand{\rowfour}[4]{\left(\begin{array}{cccc}#1&#2&#3&#4\end{array}\right)}

% A 6-element row vector \rowsix{arg1}{arg2}{arg3}{arg4}{arg5}{arg6}
\newcommand{\rowsix}[6]{\left(\begin{array}{cccccc}#1&#2&#3&#4&#5&#6\end{array}\right)}

% A 7-element row vector \rowseven{arg1}{arg2}{arg3}{arg4}{arg5}{arg6}{arg7}
\newcommand{\rowseven}[7]{\left(\begin{array}{ccccccc}#1&#2&#3&#4&#5&#6&#7\end{array}\right)}

% A 3*3 matrix \matthree{arg11}{arg12}...{arg33}
\newcommand{\matthree}[9]{\left[\begin{array}{ccc}#1&#2&#3\\#4&#5&#6\\#7&#8&#9\end{array}\right]}

% A 2*3 matrix \mattwothree{arg11}{arg12}...{arg23}
\newcommand{\mattwothree}[6]{\left[\begin{array}{ccc}#1&#2&#3\\#4&#5&#6\end{array}\right]}

% A 4*4 matrix \matfour{row1}{row2}{row3}{row4}
% The arguments are rows because we can't have more than 9 arguments 
\newcommand{\matfour}[4]{\left[\begin{array}{cccc}#1\\#2\\#3\\#4\end{array}\right]}

% A 3*2 matrix \matthreetwo{arg11}{arg12}...{arg32}
\newcommand{\matthreetwo}[6]{\left[\begin{array}{cc}#1&#2\\#3&#4\\#5&#6\end{array}\right]}

% A 2*2 matrix \mattwo{arg11}{arg12}...{arg22}
\newcommand{\mattwo}[4]{\left[\begin{array}{cc}#1&#2\\#3&#4\end{array}\right]}

% A 3*3 right-justified matrix \matthreeright{arg11}{arg12}...{arg33}
\newcommand{\matthreeright}[9]{\left[\begin{array}{rrr}#1&#2&#3\\#4&#5&#6\\#7&#8&#9\end{array}\right]}

% A matrix transforming between two reference frames (so M with 2 superscripts)
% Argument is just the supersrcipt we want
\newcommand{\mtrans}[1]{\matxx{M^{#1}}} 

% A rotation transforming between two reference frames 
% (so R with 2 superscripts)
% Argument is just the supersrcipt we want
\newcommand{\rtrans}[1]{\matxx{R^{#1}}} 

% A derivative
\newcommand{\diff}[2]{ \frac{d#1}{d#2} }

% A partial derivative
\newcommand{\pdiff}[2]{ \frac{\partial#1}{\partial#2} }

% A bracketed derivative
\newcommand{\bracdiff}[2]{\left( \frac{d#1}{d#2}      \right)}


\bibliographystyle{plain}

\title{{Models and State Representation in Scene: Generalised Software for Sequential Map-Building and Localisation}
}

\author{Andrew J. Davison\\
Robotics Research Group\\
Department of Engineering Science\\
University of Oxford\\
UK\\
ajd@robots.ox.ac.uk\\
http://www.robots.ox.ac.uk/\~{ }ajd/
}

\maketitle

%\tableofcontents 

\section{Introduction}
\label{section:introduction}

A major goal of Scene is to permit the uniform use of sequential
estimation methods across the widest possible range of robot, camera
or other mobile sensor platforms. To this end, a modular {\bf model}
system is used to represent the specifics of a particular system, so
that applying Scene to a new application is a simple matter of
creating new model classes which can be ``plugged in'' to the main
system.

In this document we will discuss in general the meaning of using
mathematical models to represent real-world systems, and then look
specifically at how models are used in Scene to describe motion and
measurements in one, two and three dimensions, explaining the
procedure needed to use Scene in a new application domain by building
new model classes.

This document is best read in combination with a study of the source
code of the example programs using the Scene Library which are
supplied in the SceneApp package: the specific implementations there
will clarify the general explanations given here.

See~\cite{Davison:Kita:Smile2000} for more general information
on the Sequential Localisation and Map-Building theory behind Scene. 

\section{Models}

A model is a mathematical representation of a real-world process. If
we wish to make sense of any data obtained from measurements the real
world, a model is essential: it is what connects the numbers we put
into and get out of a system.

The way we prefer to define models, however, is as follows:
\bi
\item {\bf A model is a simplification of reality.} 
\ei
In the real world, systems are never made from pure
geometrical objects interacting in perfect ways.
Even saying something like ``the side of this object is flat'' is, if
we examine the object closely enough, an approximation to its
non-smooth structure on a microscopic or molecular scale --- while
this is a rather extreme way to think about things, we feel that it
is important to realise what is going on when we talk about models:
they are always simplifications.

Constructing a model is a process of forming a description of a
real-world system which reduces its actual complexity to equations and
a number of {\bf parameters} describing the specifics of a simplified
mathematical process. The amount of complexity in the real-world
system which we attempt to represent in the model is always a
choice. Think of building a model for an articulated robot arm: the
choice taken here would usually be to represent it as a chain of rigid
components joined by hinged couplings with motors and encoders. A more
complex model, though, might also take into account factors such as
the flexibility of the ``rigid'' components. A line must be drawn 
somewhere --- usually
in a position such that the model remains mathematically convenient
while representing reality with a high enough degree of realism to
satisfy our needs.

As well as representing what we know about a system in a model, we
must also acknowledge the finer details that we don't know about the
system: this is what is usually referred to as noise, or uncertainty
in the model.  {\bf Noise accounts for the things that we don't
attempt to model.} There is of course no such thing as random noise in
(classical!) physics: bodies move and interact deterministically. If
we had a perfect description of a system then we would not need
measurements to tell us what was happening to it: all we would need to
do would be to set up the ``perfect model'' and let it run.  However,
while in theory we could model everything (slipping wheels; human
joints, muscles, chemical reactions!), in reality we can't know all
the details of what is happening to a system. Models have to stop
somewhere: the rest of what is going on we call noise, and estimate
the size of its effect. It is this measure of the amount a model
potentially differs from reality which means that {\bf measurements}
are able to improve our knowledge of a system. Measurement processes
themselves have models with uncertainty, and thus can be combined in a
weighted sense with the estimates from our system model to get
improved estimates.


\section{Using Models in Scene}

Scene's current domain of applicability can be defined as follows:
\bi
\item Scene can be applied to any situation where we wish to estimate the
states of a generally moving robot and many stationary features of
which it is able to make measurements via one or more sensors which
are mounted in a fixed position relative to the robot.
\ei
(Note that there are several interesting ways of using Scene which fit
these criteria although they might not initially seem to do so. One is
that a moving, active sensor such as a pair of cameras mounted on a
motorised platform can be used as long as the whole setup is
considered as a single sensor (with no internal state) whose base is
fixed rigidly to the main robot. Another is that Scene has
successfully been applied to a pair of cooperating robots in the case
that only one of them carries sensors able to make measurements of
world features: the second robot can be thought of as simply an
extension to the first, in rather the same way as we would naturally
model a two-armed robot as a single system. It does not
matter that the two cooperating robots are not physically connected.)


We will often refer to the ``state'' of a system: this is the vector
of parameters which, along with the equations specifying our model,
represent what is currently known about the system. In Scene,
the total state vector $\vecx$ can be partitioned as follows:

\beq
\vecx = \vecfour{\vecx_v}{\vecy_1}{\vecy_2}{\vdots}
~, 
\eeq
where $\vecx_v$ is the state of the robot and $\vecy_i$ is the state
of the $i$th feature. Computation of how this state vector changes
over time is the goal of all the processing in Scene. The two main
operations which take place on the state vector are:
\be
\item The process equation, which describes how the state of the robot
changes over time (the motion model):
\beq
\vecx_v(t + \Delta t) = \vecf_v(\vecx_v(t), \vecu, \Delta t) + \vecq~.
\eeq
$\vecq$ is a zero-mean Gaussian noise vector with covariance $\matQ$
(the process noise).
Since the features are assumed to be stationary, their states are not
involved in the process equation. 

\item The measurement equation, which describes how a measurement is
made of a feature (the feature measurement model):
\beq
\vecz_i = \vech_i(\vecx_v, \vecy_i) + \vecr~.
\eeq
$\vecr$ is a zero-mean Gaussian noise vector with covariance $\matR$
(the measurement noise).
\ee

%Our current estimate of this true state is the vector $\vecxhat$ with
%covariance matrix $\matP$. These can be written in the partitioned forms:
%\beq
%\vecxhat = \vecfour{\vecxhat_v}{\vecyhat_1}{\vecyhat_2}{\vdots} 
%~~,~~
%\matP = \matfour{\matP_{xx} & \matP_{xy_1} & \matP_{xy_2} & \ldots}
%{\matP_{y_1x} & \matP_{y_1y_1} & \matP_{y_1y_2} & \ldots}
%{\matP_{y_2x} & \matP_{y_2y_1} & \matP_{y_2y_2} & \ldots}
%{\vdots       & \vdots         & \vdots         & }
%~.
%\eeq
%The partitioned sections of the covariance matrix represent the
%coupling of the uncertainties in the different parts of the estimated state.
%$\matP_{xy_i}$ is the covariance matrix between the estimated robot state
%$\vecxhat_v$ and that of one of the features
%$\vecyhat_i$, while $\matP_{y_iy_j}$ is that between the estimated
%locations of two features $\vecyhat_i$ and $\vecyhat_j$.

Which parameters belong in the state vector? Scene allows a free
choice to be made about what to choose as the representation for the
robot state $x_v$ and that of each feature type state $y_i$. As
discussed above, parameters are numbers which describe the specifics
of a mathematical model of a system. However, the equations of a model
might have parameters which need not be in the state vector: they can
be treated as constant, for instance, like the length of the jointed
sections of a robot arm.  On a first reflection then, one might define
state parameters as ``the parameters that change over time.''

However, sometimes it is worthwhile to include in a state vector a
parameter which represents something in the system which is expected
to remain constant. While that real quantity stays constant, our
estimate of it can change (in fact it can only improve) over time as more
measurements are made. This if a form of on-line self-calibration.

As an example, in previous work on robot localisation it was found that our
model included one particularly significant systematic error: when a
velocity command was sent to the robot, the actual velocity achieved
was not normally distributed about the commanded value, but actually
about some other mean which differed significantly; e.g. when
sending the command $1ms^{-1}$ the robot was actually running at $0.9
\pm 0.1 ms^{-1}$ rather than $1.0 \pm 0.1 ms^{-1}$: there was a
scaling between the two which was not equal to unity. To overcome this
problem, we included in the state vector a parameter representing the
scaling between velocity commands sent and the real velocity of the
robot. This is not something which expected to change over time;
rather it is a constant characteristic of a particular robot about
which we are not very certain. Therefore, taking a first guess of the
parameter's value (e.g. 1), and a measure of the standard deviation of
that initial guess (e.g. 0.2), we initialise the parameter into the
state vector with non-zero covariance. Although the parameter does not
get changed in the model's process equation, its involvement in
various equations and Jacobians means that over time and with
measurements its estimated value will change and its uncertainty will
decrease. The estimate should converge towards the true value.

In general, if we write down equations to model a system, {\bf any
parameter which appears in the equations could be placed in the state vector.}
Parameters which are placed in the state vector will have their
estimates improved by measurements, while those left out will be be
constants whose estimated value will not change. A choice must be made
as to which parameters should be placed in the state vector. The
downside of including many parameters is that this causes
computational complexity to be increased. However, if the value of a
parameter is relatively uncertain, it is worth considering including it
in the state vector.

A final third type of parameter concerns the effect of {\bf known
outside influences} on the modelled system. Since physics is
deterministic, a model of any enclosed system will be deterministic:
it will be possible to predict its state all future times. However,
when we model a real object such as a robot, we do not usually have a
truly enclosed system: there may be something which is
influencing it, for instance via purposive commands and inputs, whether human operator or
onboard computer. Whichever it is, its effect cannot be included in a
deterministic model: we do not try to represent the state of the
computer or a human brain in our state vector; but its influence is
not constant either. Nevertheless, we may have information about the
inputs the robot is receiving and these affect how the robot's state
will evolve. We call these input parameters {\bf control parameters}
and in equations use the vector {\vecu} to refer to them.

Note: there may also be various {\bf unknown outside influences} on a
system. In this case, their effect must be included as noise in the
process equation (if the outside influences are truly unknown then
they have a ``zero mean'' overall effect so do not appear in the
process equation itself). For instance, if instead of a robot we have
mounted a sensor on a human head (such as a forehead-mounted,
outward-looking camera) and aim to estimate the head's motion in the
same way as which we would estimate the motion of a robot equipped
with the same sensor, the way in which the
person moves is unknown, but contributes uncertainty to the camera's
motion in a way which can be modelled. Again, all we are doing here is
accepting that noise in a model represents the things we don't attempt
to model.

Summarising, parameters can be divided into three types:
\be
\item {\bf State parameters}, which are parameters describing moving
aspects of the system, or stationary aspects of which we wish to
improve initially uncertain estimates.

\item {\bf Constant parameters}, of which we have an estimate with
which we are happy and which we are content not to improve.

\item {\bf Control parameters}, which are quantities which
affect the future state of the model but which are not known; they
come from outside the system under consideration, such as control
inputs to a robot. They are like one-off constants which appear in the
system equations transiently.
\ee


\section{Position State}

We introduce here the concept of {\bf position state} $\vecx_p$: this
is a standard, minimal way to represent the raw geometrical position
and pose of a robot in 1, 2 or 3D space.

Most implementations of robot localisation have made no distinction
between the concepts of state and position state: usually, what we
would call the position state --- the minimal description of robot
location --- is what goes directly into the state vector.  However, as
discussed above, modelling systems sometimes calls for parameters
additional to those purely describing position to be part of the state
vector --- in systems with redundant degrees of freedom for example.
While it may be the case that the additional parameters appear in the
state vector simply in addition to those representing pure position,
more generally the position state is something which is functionally
derived from all the state parameters:

\beq
\vecx_p = \vecx_p(\vecx_v)
\eeq

For instance, if our system is a robot arm, the parameters which we
store in the state vector might be the angles of articulation at each
joint. A sensor mounted at the end of the arm has 3D position
which is a rather complicated function of all of these angles and the
constant characteristics of the arm.


In Scene, we gain greatly from the concept of position state because
it allows {\bf separation} of the details of a particular robot motion
model from the essential position information which allows that object
to interact with other object in the world. Specifically, we can
separate the concepts of motion model and feature measurement model.


When specifying a feature measurement model, we need to define
functions such as $\vech_i(\vecx_v, \vecy_i)$ (the measurement
obtained of a feature as a function of the robot state and the
feature's state). By abstracting this to $\vech_i(\vecx_p, \vecy_i)$
(the measurement as a function of {\bf robot position} and feature
state), where we know from the definition of position state
$\vecx_p(\vecx_v)$, we are able to use measurement models which are
decoupled from specific robot motion models. For instance, we could
use the same model of a camera whether it is fixed rigidly to a robot
vehicle or mounted at the end of an articulated robot arm.


It should be noted that the concept of position state, while
convenient at the current time, is only a step towards what we
would like to achieve with the future design of Scene, in which a
robot models could have multiple ``hooks'' into position-state-like
quantities. This would be necessary if, for instance, a robot
consisted of multiple non-rigid parts, more than one of which was
equipped with a sensor.

\subsection{Position State in One Dimension}

\beq
\vecx_p = (z)
\eeq

In 1D, the position state consists of a single parameter $z$
describing displacement along a straight line.

\subsection{Position State in Two Dimensions}

\beq
\vecx_p = \vecthree{z}{x}{\phi}
\eeq

\begin{figure}[t]
\centerline{
\psfig{file=2d.id,width=50mm}
}
\caption{\label{fig:2d}
Coordinate frames in two dimensions.
}
\vspace{2mm} \hrule
\end{figure}

In 2D movement, position on a plane can be specified by 3 parameters:
the cartesian coordinates $z$ and $x$ and angle $\phi$ (restricted to
the range $-\pi < \phi <= \pi$) representing orientation relative to
the $z$ axis. We apologise
somewhat here for our choice of $z$ and $x$ as our 2D coordinates
rather than the more usual $x$ and $y$ --- this choice stems from our
original application domain in computer vision, where traditionally
the $z$ axis of coordinate frames are aligned with the optic axis of
cameras: for a robot with a forward-facing camera, this makes $z$
horizontal. See Figure~\ref{fig:2d} for clarification of this.

\subsection{Position State in Three Dimensions}


Representing position and pose in 3D is substantially more complicated
than the 1D and 2D cases. There are several different ways to
represent 3D orientation, but each has its disadvantages. Minimally,
three parameters are needed to represent 3D orientation (as in for
example the Euler angle parameterisation), with a further three
required for cartesian position, leading to a total of six
parameters. However, we have chosen to add one parameter to this
minimal six for our standard representation of 3D position and pose
and use a {\bf quaternion} of 4 parameters to represent orientation.

\beq
\vecx_p = \left(\begin{array}{c}x\\y\\z\\q_0\\q_x\\q_y\\q_z\end{array}\right)
\eeq

Quaternions are a well-established way to represent 3D orientation.
Any 3D rotation can be described by a single rotation about an
appropriately placed axis. In a quaternion, a unit vector $\vecu$
(with elements $u_x$, $u_y$, $u_z$) representing the axis of this
rotation and its angular magnitude $\theta$ are stored as follows:
\beq
\vecfour{q_0}{q_x}{q_y}{q_z} = \vecfour{\cos \frac{\theta}{2}}{u_x\sin \frac{\theta}{2}}{u_y\sin \frac{\theta}{2}}{u_z\sin \frac{\theta}{2}} ~.
\eeq
Quaternions are defined in this way because they have an algebra which
allows rotations to be conveniently composed (applied in sequence). 
In Scene, in the 3D position state vector a quaternion represents the
rotation of the robot with respect to the fixed world coordinate frame.
The following is a summary of the relevant properties of quaternions:

\bi
\item The magnitude of a quaternion, defined as the square root of the
sum of the squares of the elements, is always 1:
\beq
q_0 ^ 2 + q_x ^ 2 + q_y ^ 2 + q_z ^ 2 = 1^2~.
\eeq 
\item Quaternion $\vecq$ has a conjugate $\vecqbar$ which represents a
rotation about the same axis but with negative magnitude, and which is
defined as:
\beq
\vecqbar = \vecfour{q_0}{-q_x}{-q_y}{-q_z}~.
\eeq
\item The rotation matrix $\matR$ associated with quaternion $\vecq$
is defined as follows:
\beq
\matR \vecv = \vecq \times \vecv \times \vecqbar
\eeq
where $\vecv$ is an arbitrary $3 \times 1$ column vector. We can
calculate that:
\beq
\matR = \matthree{q_0 ^ 2 + q_x ^ 2 - q_y ^ 2 - q_z ^ 2}{2(q_x q_y -
q_0 q_z)}{2(q_x q_z + q_0 q_y)}
{2(q_x q_y + q_0 q_z)}{q_0 ^ 2 - q_x ^ 2 + q_y ^ 2 - q_z ^ 2}{2(q_y q_z - q_0 q_x)}
{2(q_x q_z - q_0 q_y)}{2(q_y q_z + q_0 q_x)}{q_0 ^ 2 - q_x ^ 2 - q_y ^ 2 + q_z ^ 2}~.
\eeq
So, if we have quaternion $\vecq$ as part of the
position state of a robot and form rotation matrix $\matR$ as above, 
this $\matR$ will relate
vectors in the {\bf fixed world coordinate frame} $W$ and the {\bf robot
coordinate frame} $R$ (which is a coordinate frame carried around by
the robot) as follows:
\beq
\vecv^W = \matR \vecv^R ~.
\eeq
We prefer to be clear in our descriptions of rotation matrices, and
refer to this $\matR$ as $\matR^{WR}$ since it relates vectors in
frames $W$ and $R$: the equation becomes:
\beq
\vecv^W = \matR^{WR} \vecv^R ~.
\eeq
Note that it is very difficult to get confused about rotation matrices
when this notation is used: in $\vecv^W = \matR^{WR} \vecv^R$ the
two frame suffices of the rotation matrix are on the sides closest to
the vector specified in those frames.

(When referring to vectors, we use the plain notation $\vecv$ to mean
a purely spatial vector, a directed segment in space not associated
with any coordinate frame. The notation $\vecv^A$ means the vector of
coordinate parameters representing that spatial vector in frame $A$.)

\item Composing rotations: if quaternion $\vecq_1$ represents the
rotation $\matR^{AB}$ and quaternion $\vecq_2$ represents the
rotation $\matR^{BC}$, then the composite rotation $\matR^{AC} =
\matR^{AB} \matR^{BC}$ is represented by the {\em product} of the two
quaternions, defined as:
\beq
\vecq_3 = \vecq_1 \times \vecq_2 = 
\left(
\begin{array}{c}
q_{10}q_{20} - (q_{1x}q_{2x} + q_{1y}q_{2y} + q_{1z}q_{2z})\\
q_{10}\vecthree{q_{2x}}{q_{2y}}{q_{2z}} +
q_{20}\vecthree{q_{1x}}{q_{1y}}{q_{1z}} +
\left(
\begin{array}{c}
q_{1y}q_{2z} - q_{2y}q_{1z}\\
q_{1z}q_{2x} - q_{2z}q_{1x}\\
q_{1x}q_{2y} - q_{2x}q_{1y}
\end{array}
\right)
\end{array}
\right)
\eeq

\ei

The disadvantage of the quaternion representation is that it is
redundant, using four parameters to represent something which can be
described minimally with three. This leads to the minor factor of
extra computational complexity in calculations due to the extra
parameter, but more importantly it means that care must be taken to
make sure that the four parameters in the quaternion part of the state
vector represent a true quaternion: that is to say that they satisfy
the magnitude criterion above.

For this reason, generalised normalisation functionality has been
built into the models in Scene: a model class should know how to
normalise its state vector if necessary. In a model containing a
quaternion, this involves enforcing the magnitude constraint described
above.

\section{Model Classes}

In Scene, modularity is implemented via a system of {\tt Model}
classes. There are three types:
\be
\item {\tt Motion\_Model} classes describing the movement of a
robot system of other sensor platform. 
\item {\tt Feature\_Measurement\_Model} classes each describing the
process of measuring a feature with a particular sensor. 
\item {\tt Internal\_Measurement\_Model} classes describing
measurements that a system makes of itself.
\ee

{\tt Motion\_Model}, {\tt Feature\_Measurement\_Model},
{\tt Internal\_Measurement\_Model} base classes are defined in the source file
{\tt SceneLib/Scene/models\_base.h} within the SceneLib
distribution. When building a Scene application, specific model
classes are derived from these base classes. The derived classes will
share the interfaces of their parents but of course provide specific
instatiation of the functionality.


In a program using Scene, each specific model class to be used is
instantiated only once and then a pointer to that class is passed
around as needed. There will be just one motion model,
zero, one or several feature measurement models (several if there is more
than one type of feature of which it is possible to make
measurements), and then optionally one internal measurement model. (In
future versions of Scene it is planned to increase generality by
providing greater choice here.) The specific model classes need to be 
defined by the creator of a new Scene application, and then
instantiated in a program so that pointers can be passed to Scene's
main classes which will make use of the models' functions to perform
calculations.


Functions in model classes are used in a uniform way. A function named {\tt
func\_A\_and\_B(C, D)} calculates the result vectors or matrices {\tt
A} and {\tt B} from input vectors or matrices {\tt C} and {\tt D}. The
results will be stored in pre-allocated spaces {\tt CRES} and {\tt DRES}
in the model classes, from which they should be copied promptly after
calling the function.

Several of the functions in model classes calculate Jacobians. A
Jacobian is a matrix of derivatives of one vector with respect to the
other. If vector $\veca(\vecb)$ is a function of vector $\vecb$,
the Jacobian $\pdiff{\veca}{\vecb}$ is defined as follows:
\beq
\pdiff{\veca}{\vecb} = \matfour{\pdiff{a_1}{b_1} & \pdiff{a_1}{b_2}  & \pdiff{a_1}{b_3}  & \ldots}
{\pdiff{a_2}{b_1}  &\pdiff{a_2}{b_2}   &\pdiff{a_2}{b_3}   & \ldots}
{\pdiff{a_3}{b_1}  & \pdiff{a_3}{b_2}  &\pdiff{a_3}{b_3}   & \ldots}
{\vdots       & \vdots         & \vdots         & }
\eeq

In the following detailed look at each of the types of model class,
vectors and matrices are named as they appear in plain text form
within the source code of Scene: thus $\vecx_v$ for example becomes
{\tt xv}.

\subsection{Defining a Motion Model}

Two levels of derivation are used with motion models. From the base
class {\tt Motion\_Model}, motion model classes specific to a
particular dimensionality of movement {\tt OneD\_Motion\_Model}, {\tt
TwoD\_Motion\_Model} are {\tt ThreeD\_Motion\_Model} are
derived (again defined in the file {\tt
SceneLib/Scene/models\_base.h}). One of these should then be used as the parent for a specific
motion model to describe a particular system.

In a specific motion model class, the following constants must be defined:

\be
\item {\tt STATE\_SIZE}, the integer state vector size.
\item {\tt CONTROL\_SIZE}, the integer control vector size.
\ee
And the following functions:
\be
\item {\tt func\_fv\_and\_dfv\_by\_dxv(xv, u, delta\_t)}, the main motion
model function (process equation) which calculates a new state {\tt fv} as a function of the old
state {\tt xv}, control vector {\tt u} and time interval {\tt
delta\_t}. Also calculates the Jacobian {\tt dfv\_by\_dxv}.
\item {\tt func\_Q(xv, u, delta\_t)}, which forms the covariance
matrix {\tt Q} of the process noise associated with {\tt fv}.
\item {\tt func\_xp(xv)}, the function which defines the position
state {\tt xp} in terms of {\tt xv}.
\item {\tt func\_dxp\_by\_dxv(xv)} which is the Jacobian for {\tt func\_xp}.
\item {\tt func\_fv\_noisy(xv\_true, u\_true, delta\_t)}, a noisy
version of the process equation which is used in simulation only to 
This function should follow the same basic form as
{\tt func\_fv\_and\_dfv\_by\_dxv} but incorporate random noise to represent
uncertainty in the model.
\item {\tt
func\_xvredef\_and\_dxvredef\_by\_dxv\_and\_dxvredef\_by\_dxpdef(xv,
xpdef)} specifies how to redefine the robot state {\tt xv} to {\tt
xvredef} when axes are redefined such that {\tt xpdef}, a given
position state, becomes the new zero of coordinates.
\ee

These should optionally be defined depending on the system:

\be
\item {\tt func\_xvnorm\_and\_dxvnorm\_by\_dxv(xv)}, a function which
defines how to {\bf normalise} the parameters in {\tt xv}, forming a
new state {\tt xvnorm}.
\item {\tt navigate\_to\_waypoint(xv, xv\_goal, u, delta\_t)}, which
can be used in systems with control parameters to set them
automatically with the aim of reaching a goal state {\tt xv\_goal}.
\ee


\subsection{Defining a Feature Measurement Model}

With feature measurement models, two levels of derivation are again
used. From the base class {\tt Feature\_Measurement\_Model}, models
specific to particular geometrical feature types are derived like {\tt
Point\_Feature\_Measurement\_Model} and {\tt
Line\_Feature\_Measurement\_Model} (in {\tt
SceneLib/Scene/models\_base.h)}. Then a specific feature measurement
model class should be derived from one of these types.

In a specific feature measurement model class, the following constants need to
be defined:
\be
\item {\tt MEASUREMENT\_SIZE}, the number of parameters representing a
measurement of the feature.
\item {\tt FEATURE\_STATE\_SIZE}, the number of parameters to
represent the state of the feature.
\ee

And the following functions:

\be

\item {\tt func\_yipose\_and\_Pyiyipose(yi, Pyiyi)}, a function which
takes the state and covariance of the feature and calculates its {\bf
pose state}: this is the basic representation of that geometrical
feature type. This function is used for graphics only.
\item {\tt func\_yi\_and\_dyi\_by\_dxp\_and\_dyi\_by\_dhi\_and\_Ri(hi,
xp)}, the feature initialisation function and its Jacobians, which
calculates a feature's state {\tt yi} from a measurement {\tt hi}
and the robot position state {\tt xp}.
\item {\tt func\_hi\_and\_dhi\_by\_dxp\_and\_dhi\_by\_dyi(yi, xp)},
the main measurement function and its Jacobians, which calculates a
measurement {\tt hi} from the feature state {\tt yi} and the robot
position state {\tt xp}.
\item {\tt func\_Ri(hi)}, which calculates the covariance {\tt Ri} of
the measurement noise for measurement {\tt hi}.
\item {\tt func\_nui(hi, zi)}, a function which calculates the
innovation {\tt nui} of a measurement with predicted value {\tt hi}
and actual value {\tt zi}. Normally this function should  perform the
simple subtraction {\tt nui} $=$ {\tt zi} $-$ {\tt hi} but the
function is left as user-definable because in some cases that is not
the case: for instance, if a measurement parameter is an angle in the
range $-\pi \rightarrow \pi$, cases where {\tt zi} and {\tt hi} lie
either side of $\pi$ would give an incorrectly large innovation if a
simple subtraction was performed: in this case the user-defined
innovation function can normalise the angle to the $-\pi \rightarrow
\pi$ range.
\item {\tt func\_hi\_noisy(yi\_true, xp\_true)}, a noisy measurement
function for use in simulation, producing a measurement with random
noise added.
\item {\tt
func\_zeroedyi\_and\_dzeroedyi\_by\_dxp\_and\_dzeroedyi\_by\_dyi(yi,
xp)}, a function which defines how to redefine the state of a feature
in the case of a redefinition of axes. The position state {\tt xp} at
which the axes are to be redefined is given, and a new feature state
{\tt zeroedyi} is calculated from the current value {\tt yi}, plus the
relevant Jacobians.
\item {\tt visibility\_test(xp, yi, xp\_orig, hi)}, which is used in
active selection of measurements to decide whether the feature should
be attempted to be measured from robot position {\tt xp}. As well as
the feature state {\tt yi} and current predicted measurement {\tt hi},
the position state {\tt xp\_orig} of the robot when the
feature was first observed is passed. A criterion for the
expected measurability of the feature should be defined based on these
relative positions: for instance, if with vision we are attempting to match a
template patch via image correlation matching, success could only be expected from
within a limited range of robot motion away from the position from
which the feature was first seen and the template stored.
This function is different from other model functions in that it
returns a single integer value representing success (0) or failure
(other value).
\item {\tt selection\_score(Si)} is a function which calculates a
score for a feature representing its value for immediate measurement
based on the innovation covariance {\tt Si}. This criterion will be
used within Scene to compare candidate measurements of
different features and allow resources to be devoted to where they are
most useful. In general, measurements with a high {\tt Si} should be
favoured because it makes sense to  make a measurement of a feature where
the result is uncertain rather than of one where it is possible to
accurately predict the result.

\ee

\subsection{Defining an Internal Measurement Model}

The derivation hierarchy for internal measurement models does not have
any intermediate steps and all specific models should be derived
directly from the base class {\tt Internal\_Measurement\_Model}. 
The constants and functions which must be defined for an internal
measurement model are very similar to those for a feature measurement
model. Note however, the important difference that while a feature
measurement model depends only on robot position state and is thus
potentially compatible with many different motion models, an internal
measurement model is specific to a particular motion model.

The following constant must be defined:

\be
\item {\tt MEASUREMENT\_SIZE}, the number of parameters in the
measurement vector.
\ee

And the following functions:

\be
\item {\tt func\_hv\_and\_dhv\_by\_dxv(xv)}, the measurement function
calculating measurement {\tt hv} and Jacovian {\tt dhv\_by\_dxv} from
the robot state {\tt xv}.
\item {\tt func\_Rv(hv)}, which calculates measurement noise {\tt Rv}
from predicted measurement {\tt hv}.
\item {\tt func\_nuv(hv, zv)}, which calculates the innovation {\tt
nuv} from predicted measurement {\tt hv} and actual measurement {\tt zv}.
\item {\tt func\_hv\_noisy(xv\_true)}, the noisy measurement function
for use in simulation.
\item {\tt feasibility\_test(xv, hv)}, which calculates the 
feasibility of the measurement based on the current robot state {\tt
xv} and predicted measurement {\tt hv}. Analogous to function {\tt
visibility\_test} in feature measurement models, this function returns
0 if the measurement is possible.
\ee

\section{Conclusion}

We have discussed the meaning of making mathematical models of
real-world systems, and shown how model classes should be defined for
use in Scene. This document will evolve as the design of Scene
changes, hopefully towards a more powerful generic modelling framework
supporting systems where multiple moving objects can be equipped with
multiple independent sensors.

\bibliography{bib}
\end{document}
